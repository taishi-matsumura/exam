% Intended LaTeX compiler: pdflatex
\documentclass[11pt]{article}
\usepackage[utf8]{inputenc}
\usepackage[T1]{fontenc}
\usepackage{graphicx}
\usepackage{grffile}
\usepackage{longtable}
\usepackage{wrapfig}
\usepackage{rotating}
\usepackage[normalem]{ulem}
\usepackage{amsmath}
\usepackage{textcomp}
\usepackage{amssymb}
\usepackage{capt-of}
\usepackage{hyperref}
\date{\today}
\title{1年生 後期 DirectX試験(第1回)}
\hypersetup{
 pdfauthor={Taishi MATSUMURA},
 pdftitle={1年生 後期 DirectX試験(第1回)},
 pdfkeywords={},
 pdfsubject={},
 pdfcreator={Emacs 26.1 (Org mode 9.1.14)}, 
 pdflang={Ja}}
\begin{document}

\maketitle

\section*{問1}
\label{sec:org56aeac4}
\subsection*{その1}
\label{sec:orgb7bb1db}
4×4の単位行列を書け\\

\subsection*{その2}
\label{sec:org0dd9955}
3次元の情報(各ジオメトリの座標)を2次元の情報(スクリーン上の座標)に変換する過程で必要な変換処理3種類を書け\\

\subsection*{その3}
\label{sec:orge1ec434}
以下の機能はどの変換処理を用いて実現するか、それぞれについて答えよ\\
\begin{enumerate}
\item ジオメトリの回転\\
\item ジオメトリの拡大・縮小\\
\item ジオメトリの平行移動\\
\item カメラの回転\\
\item カメラの移動\\
\item ズームイン・ズームアウト\\
\end{enumerate}


\section*{問2}
\label{sec:org7cdda78}
\subsection*{その1}
\label{sec:org7a0a377}
以下は全て「D3DXMatrix」で始まるヘルパー関数である。それぞれの関数名を答えよ(D3DXMatrixは書かなくて良い)\\
\begin{enumerate}
\item 単位行列作成\\
\item 行列の積\\
\item 平行移動行列作成(オフセット)\\
\item X軸回転行列作成\\
\item ビュー変換行列作成\\
\item 射影変換行列作成\\
\end{enumerate}

\subsection*{その2}
\label{sec:org816cb46}
その1の5.の関数を使用する際に必要なパラメータを3つ書け。\\


\section*{問3}
\label{sec:org1c60f83}
以下は2Dの回転行列の求め方について書かれている。空白部を埋めよ\\

xy平面上の点(x, y)を原点(0, 0)を中心にθ回転した点(x', y')について考える。\\

原点と(x, y)を通る直線とx軸のなす角をα、(x, y)と原点(0, 0)との距離をrとすると、\\

\begin{quote}
\begin{eqnarray}
x' = \boxed{\vphantom{0}\hspace{5em}}\\
y' = \boxed{\vphantom{0}\hspace{5em}}
\end{eqnarray}
\end{quote}

また、加法定理より、\\

\begin{quote}
\begin{eqnarray}
\cos (\alpha + \theta) = \cos\alpha \cdot \cos\theta - \sin \alpha \cdot \sin \theta\\
\sin (\alpha + \theta) = \sin\alpha \cdot \cos\theta + \cos \alpha \cdot \sin \theta
\end{eqnarray}
\end{quote}

x'について整理すると\\

\begin{quote}
\begin{verbatim}
複数行のテキストをそのまま出力
この環境内で改行すると
このように反映されます。
記号も → \verb*| ! " $ % $ & |
\end{verbatim}
\end{quote}

\begin{quote}
\begin{eqnarray}
x' &=& \boxed{\vphantom{0}\hspace{5em}} \verb| (∵(1)・(3)より) |\\
   &=& \boxed{\vphantom{0}\hspace{5em}} (\because \cos \α = x/r、\sin α = y/r)\\
   &=& \boxed{\vphantom{0}\hspace{5em}}
\end{eqnarray}
\end{quote}

\begin{quote}
x' = <span class="bb"> </span>  (∵①・③より)\\
   = <span class="bb"> </span>  (∵cosα = x/r、sinα = y/r)\\
   = <span class="bb"> </span>\\

同様にy'について整理すると\\

y' = <span class="bb"> </span>  (∵②・④より)\\
   = <span class="bb"> </span>  (∵sinα = y/r、cosα = x/r)\\
   = <span class="bb"> </span>\\

xy平面上の座標(x, y)を原点(0, 0)からのベクトルと考え、それを行列で表現すると\\
\(\left(x\;y\right)\)\\
となり、この行列に2×2行列を掛けあわせたものは以下のようになる。\\
\begin{eqnarray*}
\begin{pmatrix}
x & y
\end{pmatrix}
\cdot
\begin{pmatrix}
a & c\\
b & d
\end{pmatrix}
 = 
\boxed{\phantom{hogehoge}}
\end{eqnarray*}
…⑤\\

ここで、前述の「点(x, y)を原点を中心にθ回転させた点(x', y')」についても同様に行列で表現すると\\

(x'  y') = <span class="bb"> </span>   …⑥\\

⑤・⑥の右辺に注目し、\\
a = cosθ\\
b = -sinθ\\
c = sinθ\\
d = cosθ\\
と置くと、\\

(x  y)・(a  c) = (x  y)・<span class="bb"> </span>\\
        (b  d)\\

= <span class="bb"> </span>\\
= (x'  y')\\

と変換できる。\\

∴点(x, y)を原点を中心にθ回転させた点(x', y')は\\

(x  y)・<span class="bb"> </span>\\

にて求められる。\\
\end{quote}
\end{document}